\subsection{Example 2: Collisional-Radiative Model}
\begin{frame}[fragile]
	\frametitle{\insertsubsectionhead}
	\begin{itemize}
		\item CRANE was primarily designed to solve chemical reaction networks in plasmas, but it is equally capable of solving general ODE systems as well.
		\item This example shows a simplified three-level hydrogen system for assumed sets of rate coefficients \footnotemark
		\item The model is cast into an ODE system of three mass fractions with six excitation, deexcitation, and ionization frequencies
	\end{itemize}
	\footnotetext[1]{Rehman, T. (2018). ``Studies on plasma-chemical reduction." Eindhoven: Technische Universiteit Eindhoven.}
\end{frame}

\begin{frame}[fragile]
	\frametitle{\insertsubsectionhead}
	\begin{center}
		\includegraphics[scale=0.4]{pics/hydrogen_system.png}
	\end{center}
%		\caption{The three hydrogen system with transition frequencies labeled. Figure adapted from [1]. }
	\vspace{-1em}
	
	\begin{itemize}
		\item The system corresponds to three electron-impact reactions: 
		\begin{align*}
			H + e  &\leftrightarrow H^* + e \\
			H* + e  &\leftrightarrow H^+ + e + e \\
			H + e  &\leftrightarrow H^+ + e + e 
		\end{align*}
	\end{itemize}
\end{frame}

	
	
	
\begin{frame}[fragile]
	\frametitle{\insertsubsectionhead}
	\begin{center}
	\begin{scriptsize}
	\begin{tabular}{lll}
	\hline
	\multicolumn{2}{r}{$s^{-1}$} \\
	\cline{2-3}
	Frequency  & Stiff & Nonstiff \\
	\hline
	$f_{12}$ & $2.7 \times 10^{10}$ & $9 \times 10^{1}$ \\
	$f_{13}$ & $9.0 \times 10^8$ & $1 \times 10^{2}$ \\
	$f_{23}$ & $1.0 \times 10^6$ & $5 \times 10^{1}$ \\
	$f_{32}$ & $7.5 \times 10^{4}$ & $3 \times 10^{1}$ \\
	$f_{21}$ & $3.8 \times 10^{1}$ & $1 \times 10^{1}$ \\
	$f_{31}$ & $1.7 \times 10^{2}$ & $2 \times 10^{1}$ \\
	\hline
	\end{tabular}
	\end{scriptsize}
	\end{center}
	\begin{itemize}
		\item The system may be cast from a system of three reactions (excitation, deexcitation, and ionization) into an ODE system:
		\begin{scriptsize}
		\begin{minipage}{0.5\linewidth}
		\begin{align*}
			&\frac{dy_1}{dt} = -(f_{12} + f_{13})y_1 + f_{21}y_2 + f_{31}y_3 \\
			&\frac{dy_2}{dt} = f_{12}y_1 - (f_{23} + f_{21})y_2 + f_{32}y_3 \\
			&\frac{dy_3}{dt} = f_{13}y_1 + f_{23}y_2 - (f_{31} + f_{32})y_3
		\end{align*}
		\end{minipage}%
		\begin{minipage}{0.5\linewidth}
		\begin{align*}
			&y_1 \xrightarrow{f_{12}} y_2 \\
			&y_1 \xrightarrow{f_{13}} y_3 \\
			&y_2 \xrightarrow{f_{21}} y_1 \\
			&y_3 \xrightarrow{f_{31}} y_1 \\
			&y_2 \xrightarrow{f_{23}} y_3 \\
			&y_3 \xrightarrow{f_{32}} y_2 
		\end{align*}
		\end{minipage}
		\end{scriptsize}
	\end{itemize}
\end{frame}

\begin{frame}[fragile]
	\frametitle{\insertsubsectionhead \hspace{1pt} - Running}
		\begin{itemize}
			\item The problem input file is located in the \texttt{problems} directory, named \texttt{collisional\_radiative.i}
			\item To run, navigate to the problems directory and provide the input file to the executable:
			\begin{enumerate}
				\item Navigate to the problems directory: \newline
				\hspace*{8pt} \texttt{cd \textasciitilde/projects/crane/problems}
				\item Locate the input file: \newline
				\hspace*{8pt} \texttt{problems/collisional\_radiative.i}
				\item Run the problem with CRANE: \newline
				\hspace*{8pt} \texttt{../crane-opt -i collisional\_radiative.i}
				\item Check output files: \newline
				\hspace*{8pt} \texttt{problems/collisional\_radiative\_out.csv}
			\end{enumerate}
			\item Both the stiff and nonstiff cases were run with five different initial concentrations ($y_0$, $y_1$, $y_2$): 
			\begin{enumerate}
				\item T1: (1.0, 0, 0)
				\item T2: (0.2, 0.05, 0.75)
				\item T3: (0.5, 0.2, 0.3)
				\item T4: (0.4, 0.4, 0.2)
				\item T5: (0.4, 0.6, 0)
			\end{enumerate}
		\end{itemize}
\end{frame}

\begin{frame}[fragile]
	\frametitle{\insertsubsectionhead \hspace{1pt} Results}
	\begin{minipage}{0.5\linewidth}
		\includegraphics[scale=0.4]{pics/stiff_transition.png}
	\end{minipage}%
	\begin{minipage}{0.5\linewidth}
		\includegraphics[scale=0.4]{pics/nonstiff_transition.png}
	\end{minipage}
	\begin{itemize}
		\item The figures show the phase space in both the stiff and nonstiff cases.
		\item Starting at several different initial conditions, the trajectories end up at the theoretical equilibrium points
		\begin{enumerate}
			\item Stiff: [$6.62e-10$, $0.00698$, $0.930$]
			\item Nonstiff: [$0.0769$, $0.385$, $0.538$]
		\end{enumerate}
	\end{itemize}
\end{frame}

%\begin{frame}[fragile]
%	\frametitle{\insertsubsectionhead}
%	\begin{itemize}
%		\item In order to be solved by CRANE, the problem must be cast into a system of ``reactions": 
%	\end{itemize}
%	\begin{minipage}{0.5\linewidth}
%		\begin{align*}
%			&\frac{dx}{dt} = ax - bxy \\
%			&\frac{dy}{dt} = -cy + dxy
%		\end{align*}
%	\end{minipage}%
%	\begin{minipage}{0.5\linewidth}
%		\begin{align*}
%			x &\xrightarrow{a} 2x \\
%			x + y &\xrightarrow{b} y \\
%			y &\xrightarrow{c} z \\
%			x + y &\xrightarrow{d} x + 2y
%		\end{align*}
%	\end{minipage}
%	\begin{itemize}
%		\item Problem statement:
%		\begin{align*}
%			&t = [0,50] \\
%			&x(0), \quad y(0) = 1 \\
%			&[a,b,c,d] = [\tfrac{2}{3},\tfrac{4}{3},1,1]
%		\end{align*}
%	\end{itemize}
%\end{frame}

%\begin{frame}[fragile]
%	\frametitle{\insertsubsectionhead \hspace{1pt} - Running}
%		\begin{itemize}
%			\item The problem input file is located in the  directory, named \texttt{lotka\_volterra.i}
%			\item To run, navigate to the problems directory and provide the input file to the executable:
%			\begin{enumerate}
%				\item Navigate to the problems directory: \newline
%				\hspace*{8pt} \texttt{cd \textasciitilde/projects/crane/problems}
%				\item Locate the input file: \newline
%				\hspace*{8pt} \texttt{problems/lotka\_volterra.i}
%				\item Run the problem with CRANE: \newline
%				\hspace*{8pt} \texttt{../crane-opt -i lotka\_volterra.i}
%				\item Check output files: \newline
%				\hspace*{8pt} \texttt{problems/lotka\_volterra.csv}
%				\item Plot results: \newline
%				\hspace*{8pt} \texttt{python lotka\_volterra.py}
%			\end{enumerate}
%		\end{itemize}
%\end{frame}

%\begin{frame}
%	\frametitle{\insertsubsectionhead \hspace{1pt} Results}
%	\begin{minipage}{0.5\linewidth}
%		\begin{itemize}
%			\item Outputs may be stored in either CSV or Exodus format; for scalar problems, CSV is convenient for immediate viewing and plotting
%			\item The results show a typical predator-prey system, whereby population `x' is consumed by `y', `y' becomes saturated, and `x' begins to rise again as `y' decreases
%		\end{itemize}
%	\end{minipage}%
%	\begin{minipage}{0.5\linewidth}
%	\centering
%	\includegraphics[scale=0.4]{pics/lotka_volterra.png}
%	\end{minipage}
%\end{frame}