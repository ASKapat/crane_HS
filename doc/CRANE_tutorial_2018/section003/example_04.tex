\subsection{Example 4: Microcathode Argon Discharge} 
\begin{frame}[fragile]
	\frametitle{\insertsubsectionhead}
	\begin{itemize}
		\item ZDPlasKin example 2: 13 reactions between 5 species ($Ar, Ar^+, Ar^{2+}, Ar^*, e^-$) \footnotemark
%		\begin{align*}
%			&e + Ar \xrightarrow{k_1} 2e + Ar^+  \\
%			&e + Ar^+ + Ar \xrightarrow{k_2} 2Ar  
%		\end{align*}
		\item This example shows the capability of CRANE to use expressions for both rate coefficients and other variables in the system (in this case, reduced electric field)
		\item All species are nonlinear variables
		\item Reduced electric field is calculated based on electron mobility, gas density, and physical domain
		\item Auxiliary variables: reduced electric field, mobility, current (used in reduced electric field calculation)
		\item Electron-impact rate coefficients and electron mobility calculated by \textit{Bolsig+} (results pre-computed and available in /projects/crane/problems/Example2)
	\end{itemize}
	\footnotetext[3]{{\scriptsize \color{blue} https://www.zdplaskin.laplace.univ-tlse.fr/micro-cathode-sustained-discharged-in-ar/}}
\end{frame}




\begin{frame}[fragile]
	\frametitle{Example 4: Input Parameters}
	\begin{itemize}
		\item Initial Conditions:
		\begin{itemize}
			\item[$\ast$] $e, Ar^+, Ar^*= 1e6 cm^{-3}$
			\item[$\ast$]  $Ar = 3.21883e18 cm^{-3}$ $(p = 100$ Torr $)$
			\item[$\ast$]  $Ar^{2+} = 1.0 cm^{-3}$
			\item[$\ast$] $d = 0.004 cm$, $r = 0.004 cm$, $R = 1e5 \Omega$
		\end{itemize}
		\item $T_{gas} = 300 K$ , $t=[0,1 ms]$
		\item $\Big(\tfrac{E}{N}\Big)^n = V / \Big(d + R*J/\Big(\Big(\tfrac{E}{N}\Big)^{n-1}*n_{Ar}\Big)\Big)/n_{Ar}$
		\item $J = \Big(\Big(\tfrac{E}{N}\Big)^n * \mu^n * n_{Ar}\Big) * q_e * \pi r^2 * n_e$
		\item Input file: `zdplaskin\_ex2.i'
	\end{itemize}
\end{frame}

\begin{frame}[fragile]
	\frametitle{Example 4: Calculating Parameters}
		\begin{Verbatim}[fontsize=\scriptsize]
    [./reduced_field_calculate]
      type = ParsedAuxScalar
      variable = reduced_field
      constant_names = `V d qe R'
      constant_expressions = `1000 0.004 1.602e-19 1e5'
      args = `reduced_field Ar current'
      function = `V/(d+R*current/(reduced_field*Ar*1e6))/(Ar*1e6)'
      execute_on = `TIMESTEP_END'
    [../]
	\end{Verbatim}

	\begin{itemize}
		\item Reduced electric field and current may be calculated by MOOSE as Auxiliary variables, which may be coupled into problems like any other variable
		\item `\textbf{constant\_names}' and `\textbf{constant\_expressions}' supplies the function with names and values for each constant appearing in the functions
		\item \textbf{`args'} - nonlinear or auxiliary variable values that must be accessed by function
		\item \textbf{`execute\_on'} - Tells MOOSE when to run the AuxKernel
	\end{itemize}

\end{frame}

\begin{frame}[fragile]
	\frametitle{Example 4: Reading Parameters From File}
		\begin{Verbatim}[fontsize=\footnotesize]
    [./mobility_calculation]
      type = DataReadScalar
      variable = mobility
      sampler = reduced_field
      property_file = `Example2/electron_mobility.txt'
      execute_on = `INITIAL TIMESTEP_BEGIN'
    [../]
	\end{Verbatim}

	\begin{itemize}
		\item Electron mobility and temperature are both read from data files tabulated as a function of reduced electric field (calculated by \textit{Bolsig+})
		\item The `DataReadScalar' AuxKernel may be used to pull this data from the files
		\item `property\_file' tells MOOSE where to find the tabulated data file
	\end{itemize}

\end{frame}

\begin{frame}
	\frametitle{Example 4: Results}
	\centering
	\includegraphics[width=0.75\textwidth]{./pics/example4.png}
\end{frame}