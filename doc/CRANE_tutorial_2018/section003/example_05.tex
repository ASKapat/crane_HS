\subsection{Example 5 (Hands On): Two Argon Reactions}

\begin{frame}[fragile]
	\frametitle{\insertsubsectionhead}
	%\begin{minipage}{0.6\linewidth}
	\begin{itemize}
		\item So far you have seen 4 pre-made examples of using CRANE. All of the input files are already in place, so you can ensure that the results are reproduced on your own machine(s).
		\item ...but the point of this code is to let you use it for your own applications!
		\item This next example will be \textbf{hands-on} - try to write your own input file!
		%\item The completed input file is already present for guidance: \texttt{/problems/example5.i}
		\item A ``skeleton" input file is included in \texttt{example5.i}. \textbf{You will need to add scalar variables, scalar kernels (for time derivatives), and the two reactions.}
	\end{itemize}
	%\end{minipage}%
%	\begin{minipage}{0.4\linewidth}
%		\includegraphics[scale=2.4]{pics/the_difference.png}
%	\end{minipage}
	%\footnotetext[4]{\scriptsize Image courtesy of https://xkcd.com/242/}
\end{frame}

\begin{frame}[fragile]
	\frametitle{\insertsubsectionhead}
	\begin{itemize}
		\item 2 reactions between 3 species (Ar, Ar+, e-)
		\item \textbf{ZDPlasKin example}\footnotemark: Two Reaction Test Case 
		\begin{align*}
			&e + Ar \xrightarrow{k_1} 2e + Ar^+  \\
			&e + Ar^+ + Ar \xrightarrow{k_2} 2Ar  
		\end{align*}
		\item $\tfrac{E}{N} = 50 Td \rightarrow 50\times 10^{-21} V m^2$
		\item $T_{gas} = 300 K$ , $t=[0,0.25 \mu s]$
		\item With these parameters, this becomes a system of three coupled ODEs with constant rate coefficients
		\item The reduced electric field is added as an Auxiliary variable

	\end{itemize}
	\footnotetext[5]{{\scriptsize \color{blue} https://www.zdplaskin.laplace.univ-tlse.fr/two-reaction-test-case-start-here/}}
\end{frame}

\begin{frame}[fragile]
	\frametitle{\insertsubsectionhead}
	\begin{itemize}
		\item Input Parameters:
		\begin{align*}
				&e + Ar \xrightarrow{k_1} 2e + Ar^+  \\
				&e + Ar^+ + Ar \xrightarrow{k_2} 2Ar  
			\end{align*}
		\begin{itemize}
			\item[$\ast$] $n_{Ar}(t=0) = 2.5e19 [cm^{-3}]$
			\item[$\ast$] $n_{e}(t=0), n_{Ar^+}(t=0) = 1.0 [cm^{-3}]$
			\item[$\ast$] $k_1 = $\{EEDF convolution\}
			\item[$\ast$] $k_2 = 1e-25 \quad [cm^6 s^{-1}]$
			\item[$\ast$] $t=[0,0.25 \mu s]$
		\end{itemize}
		\item \textbf{Instructions}:
		\begin{enumerate}
			\item Add scalar variables ($e^-$ is already filled in for you)
			\item Add scalar kernels (\texttt{type = ODETimeDerivative})
			\item Fill in ChemicalReactions action (2 reactions)
			\begin{itemize}
				\item Rate constant $k_1$ tabulated as a function of reduced electric field included in `projects/crane/problems/Example5/'
			\end{itemize}
		\end{enumerate}
	\end{itemize}
\end{frame}

\begin{frame}[fragile]
	\frametitle{Example 5 (Hands On): Mesh and Variables}
	\begin{minipage}{0.6\linewidth}
	\begin{itemize}
		\item First a ``dummy" mesh is needed to run a problem in MOOSE (dim = 1 with nx = 1 is sufficient)
		\item Three variables are considered in this problem: electrons, argon ions, and argon neutrals (background gas)
		\item An example of the Mesh and Variables are shown on the right. Note that you will need to add two more Variables
		\item Initial conditions:
		\begin{itemize}
			\item $e^-$: 1
			\item $Ar^+$: 1
			\item $Ar$: $2.5 \times 10^{19}$
		\end{itemize}
	\end{itemize}
	\end{minipage}%
	\begin{minipage}{0.4\linewidth}
	\begin{scriptsize}
	\begin{BVerbatim}
[Mesh]
  type = GeneratedMesh
  dim = 1
  xmin = 0
  xmax = 1
  nx = 1
[]

[Variables]
  [./e]
    family = SCALAR
    order = FIRST
    initial_condition = 1
  [../]
[]
	\end{BVerbatim}
	\end{scriptsize}
	\end{minipage}
\end{frame}

\begin{frame}[fragile]
	\frametitle{Example 5 (Hands On): Kernels and AuxVariables}
	\begin{minipage}{0.6\linewidth}
	\begin{itemize}
		\item Time derivative kernels must be added for each variable in the system. (Note that you will need to add two more to the example shown here) 
		\item One of the reaction rate coefficients is read from a tabulated file. In general these need to be computed manually by the user through BOLSIG+ or an equivalent software. 
		\item In this case, the reaction file already exists in the \texttt{/problems/Example5} directory
	\end{itemize}
	\end{minipage}%
	\begin{minipage}{0.4\linewidth}
	\begin{scriptsize}
	\begin{BVerbatim}
  [ScalarKernels]
    [./de_dt]
      type = ODETimeDerivative
      variable = e
    [../]
  []

  [AuxVariables]
    [./reduced_field]
      order = FIRST
      family = SCALAR
      initial_condition = 50e-21
    [../]
  []
	\end{BVerbatim}
	\end{scriptsize}
	
	\end{minipage}
\end{frame}

\begin{frame}[fragile]
	\frametitle{Example 5 (Hands On): Kernels and AuxVariables cont.}
	\begin{minipage}{0.6\linewidth}
	\begin{itemize}
		\item The data is tabulated with respect to the reduced electric field, $V m^{-2}$. \textbf{The sampling variable must be a variable in the system, either Nonlinear or Auxiliary.}
		\item With this in mind, we will create a (constant) \texttt{AuxVariable} called \texttt{reduced\_field}, with an initial condition of $50 \times 10^{-21} V m^{-2}$
	\end{itemize}
	\end{minipage}%
	\begin{minipage}{0.4\linewidth}
	\begin{scriptsize}
	\begin{BVerbatim}
  [ScalarKernels]
    [./de_dt]
      type = ODETimeDerivative
      variable = e
    [../]
  []

  [AuxVariables]
    [./reduced_field]
      order = FIRST
      family = SCALAR
      initial_condition = 50e-21
    [../] 
  []  
	\end{BVerbatim}
	\end{scriptsize}
	
	\end{minipage}
\end{frame}

%\begin{frame}[fragile]
%	\frametitle{Declaring Scalar Variables}
%	\begin{columns}
%	\column{0.5\textwidth}
%	\begin{Verbatim}[fontsize=\scriptsize]
%    [./Ar]
%      family = SCALAR
%      order = FIRST
%      initial_condition = 2.5e19
%      scaling = 2.5e-19
%    [../]
%	\end{Verbatim}
%	\begin{itemize}
%		\item This block declares a \textbf{first-order scalar} variable named \textbf{Ar}
%		\item The (optional) scaling parameter is used to scale the variable values closer to unity, which is desireable for convergence
%	\end{itemize}
%
%	\column{0.5\textwidth}
%	\begin{Verbatim}[fontsize=\scriptsize]
%    [./dAr_dt]
%      type = ODETimeDerivative
%      variable = Ar
%    [../]
%    
%	\end{Verbatim}
%	\begin{itemize}
%		\item All nonlinear variables need at least one Kernel active
%		\item The \textit{ChemicalReactions} Action automatically adds all reaction kernels, but the time derivative term must be added manually for each nonlinear variable
%	\end{itemize}
%	\end{columns}
%\end{frame}

%\begin{frame}[fragile]
%	\frametitle{\insertsubsectionhead: \hspace{1pt} Reactions}
%	\begin{minipage}{0.5\linewidth}
%	\begin{itemize}
%		\item The Action allows you to
%	\end{itemize}
%	\end{minipage}%
%	\begin{minipage}{0.5\linewidth}
%	\begin{scriptsize}
%	\begin{BVerbatim}
%    [ChemicalReactions]
%      [./ScalarNetwork]
%        species = 'e Ar+ Ar'
%        file_location = 'Example5'
%        sampling_format = 'reduced_field'
%
%        reactions = `e + Ar -> e + e + Ar+          : EEDF
%                     e + Ar+ + Ar -> Ar + Ar        : 1e-25'
%
%       [../]
%    []
%	\end{BVerbatim}
%	\end{scriptsize}
%	\end{minipage}
%\end{frame}

\begin{frame}[fragile]
	\frametitle{\insertsubsectionhead \hspace{1pt} - Running}
		\begin{itemize}
			\item To run, navigate to the problems directory and provide the input file to the executable:
			\begin{enumerate}
				\item Navigate to the problems directory: \newline
				\hspace*{8pt} \texttt{cd \textasciitilde/projects/crane/problems}
				\item Locate the input file: \newline
				\hspace*{8pt} \texttt{problems/example5.i}
				\item Run the problem with CRANE: \newline
				\hspace*{8pt} \texttt{../crane-opt -i example5.i}
				\item Check output files: \newline
				\hspace*{8pt} \texttt{problems/example5\_out.csv}
				\item Plot results: \newline
				\hspace*{8pt} \texttt{python example5.py}
			\end{enumerate}
		\end{itemize}
\end{frame}

\begin{frame}
	\frametitle{Example 5 Results}
	\centering
	\includegraphics[width=0.75\textwidth]{./pics/example3.png}
\end{frame}