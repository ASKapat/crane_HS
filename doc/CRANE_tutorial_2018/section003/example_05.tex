\subsection{Example 5: Nitrogen Reaction Network}
\begin{frame}[fragile]
	\frametitle{\insertsubsectionhead}
	\begin{itemize}
		\item ZDPlasKin example 3: 34 reactions between 11 species \footnotemark
		\item In this example, electric field and electron density are read from files containing time-varying data
		\item Electron temperature is calculated from \textit{Bolsig+}, as are electron-impact rate coefficients
		\item In this case, electrons \textit{are not considered a nonlinear species} and do not contribute to the jacobian
		\item Reduced electric field is calculated based on electron mobility, gas density, and physical domain
	\end{itemize}
	\footnotetext[4]{{\scriptsize \color{blue} https://www.zdplaskin.laplace.univ-tlse.fr/external-profiles-of-electron-density-and-electric-field/}}
\end{frame}




\begin{frame}[fragile]
	\frametitle{Example 5: Input Parameters}
	\begin{itemize}
		\item Initial Conditions: $N_2(t=0) = 2.447464e19$; all other species start with $n_i(0) = 0 cm^{-3}$
		\item $T_{gas} = 300 K$ , $t=[0,2.5 ms]$
		\item $T_{eff} = T_{gas}+\Big(0.12 * \Big(\tfrac{E}{N}*1e21 \Big)^2 \Big)$
		\item Reduced field, electron temperature, and electron density pulled from tabulated data 
		\item Input file: `zdplaskin\_ex3.i'
	\end{itemize}
\end{frame}

\begin{frame}[fragile]
	\frametitle{Example 5: Reading Time Data From File}
	\begin{Verbatim}[fontsize=\scriptsize]
    [./field_calculation]
      type = DataReadScalar
      variable = reduced_field
      use_time = true
      property_file = `Example3/reduced_field.txt'
      execute_on = `TIMESTEP_BEGIN'
    [../]
	\end{Verbatim}
	\begin{itemize}
		\item `DataReadScalar' AuxKernel is used as in Example 2
		\item Setting `use\_time' to `true' tells MOOSE to sample from the tabulated data set using the time variable
	\end{itemize}
\end{frame}

\begin{frame}[fragile]
	\frametitle{Example 5: Scaling Data Read From File}
	\begin{Verbatim}[fontsize=\scriptsize]
    [./field_calculation]
      type = DataReadScalar
      variable = Te
      scale_factor = 1.5e-1
      sampler = reduced_field
      property_file = `Example3/electron_temperature.txt'
      execute_on = `TIMESTEP_BEGIN'
    [../]
	\end{Verbatim}
	\begin{itemize}
		\item Electron temperature for this problem is also read from a data file, but sampled based on the `reduced\_field' AuxVariable rather than time
		\item The \textbf{scale\_factor} parameter is multiplied into the result after being read from the data file. This allows the user to convert the data to their desired units.
	\end{itemize}
\end{frame}

\begin{frame}[fragile]
	\frametitle{Example 5: Action Parameters}
	\begin{Verbatim}[fontsize=\tiny]
    [ChemicalReactions]
      [./ScalarNetwork]
        species = `e N N2 N2A N2B N2a1 N2C N+ N2+ N3+ N4+'
        aux_species = `e'
        file_location = `Example3'

        # These are parameters required equation-based rate coefficients
        equation_variables = `Te Teff'
        rate_provider_var = `reduced_field'
        reactions = `...'
      [../]
    []
	\end{Verbatim}
	\begin{itemize}
		\item Since electrons are being read from a data file, they are not a nonlinear species. The \textbf{aux\_species} parameter allows auxiliary variables to be coupled into the reaction network.
		\item Variables that are used in any equation-based rate coefficients must be listed in the \textbf{equation\_variables} parameter
	\end{itemize}
\end{frame}

\begin{frame}
	\frametitle{Example 5: Results}
	\centering
	\includegraphics[width=0.75\textwidth]{./pics/example5.png}
\end{frame}