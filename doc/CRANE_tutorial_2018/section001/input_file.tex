
	\begin{frame}[fragile]
		\frametitle{Input Structure}
		\begin{columns}
			\column{0.5\textwidth}
			\begin{itemize}
				\item A basic moose input file requires seven parts, or ``blocks":
				\begin{enumerate}
					\item Mesh
					\item Variables
					\item Kernels
					\item BCs
					\item ICs
					\item Executioner
					\item Outputs
				\end{enumerate}
				\item If no boundary condition is included for a variable, it will default to a Neumann BC
				\item Similarly, initial conditions (ICs) default to 1.
			\end{itemize}
			\column{0.5\textwidth}
			\begin{Verbatim}[fontsize=\tiny]
	[Mesh]
	  file = mug.e
	[]

	[Variables]
	  [./diffused]
	    order = FIRST
	    family = LAGRANGE
	  [../]
	[]

	[Kernels]
	  [./diff]
	    type = Diffusion
	    variable = diffused
	  [../]
	[]

	[Executioner]
	  type = Steady
	  solve_type = 'PJFNK'
	[]

	[Outputs]
	  execute_on = 'timestep_end'
	  exodus = true
	[]
			\end{Verbatim}
		\end{columns}
	\end{frame}


	\begin{frame}[fragile]
		\frametitle{Mesh}
		\begin{columns}
			\column{0.6\textwidth}
			\begin{itemize}
				\item MOOSE accepts meshes generated from third party meshing tools (e.g. Gmsh)
				\item Alternatively, MOOSE may generate a mesh based on user input
				\item The example shown here generates a 1D mesh on the domain [0,1] with 10 nodes
			\end{itemize}
			\column{0.4\textwidth}
			\begin{Verbatim}[fontsize=\large]
[Mesh]
  type = GeneratedMesh
  dim = 1
  xmin = 0
  xmax = 1
  nx = 10
[]
			\end{Verbatim}
		\end{columns}
	\end{frame}

	\begin{frame}[fragile]
		\frametitle{Variables}
		\begin{columns}
			\column{0.5\textwidth}
			\begin{itemize}
				\item The Variables block defines the nonlinear variables in the system
				\item Order options: CONSTANT, FIRST, SECOND, THIRD, FOURTH
				\item The example shown here defines a variable ``x" with an initial condition of 10.
				\item Most commonly used Families are LAGRANGE and SCALAR
				\item Other options included on the website: http://mooseframework.org/source/actions/AddVariableAction.html
			\end{itemize}
			\column{0.5\textwidth}
			\begin{Verbatim}[fontsize=\large]
[Variables]
  [./x]
    order = FIRST
    family = LAGRANGE
    initial_condition = 10
  [../]
[]
			\end{Verbatim}
		\end{columns}
	\end{frame}

	\begin{frame}[fragile]
		\frametitle{Executioner}
		\begin{columns}
			\column{0.5\textwidth}
			\begin{itemize}
				\item The Executioner block defines both the type of problem and the method with which it will be solved
				\item User may define Steady, Transient, or Eigenvalue problems
				\item Executioner block also accepts sub-blocks to define adaptive timesteppers and different integrators
			\end{itemize}
			\column{0.5\textwidth}
			\begin{Verbatim}[fontsize=\small]
[Executioner]
  type = Transient
  solve_type = `PJFNK`
  scheme = crank-nicolson
[]
			\end{Verbatim}
		\end{columns}
	\end{frame}

	\begin{frame}[fragile]
		\frametitle{Executioner cont.}
		\begin{columns}
			\column{0.5\textwidth}
			\begin{itemize}
				\item ``solve type`` defines the numerical routine used to solve the nonlinear variables at each timestep
				\item Options include: JFNK, PJFNK, NEWTON, FD, LINEAR
%				\begin{enumerate}
%					\item PJFNK (preconditioned jacobian-free Newton Krylov)
%					\item JFNK (jacobian-free Newton Krylov)
%					\item NEWTON (full Newton solve)
%					\item FD (uses finite differences to compute jacobian)
%					\item LINEAR (solves a linear problem)
%				\end{enumerate}
				\item ``scheme`` denotes the time integration scheme. Default: implicit euler. 
			\end{itemize}
			\column{0.5\textwidth}
			\begin{Verbatim}[fontsize=\small]
[Executioner]
  type = Transient
  solve_type = `PJFNK`
  scheme = crank-nicolson
[]
			\end{Verbatim}
		\end{columns}
	\end{frame}